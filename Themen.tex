\documentclass[10pt,a4paper]{article}
\usepackage[utf8]{inputenc}
\usepackage{amsmath}
\usepackage{amsfonts}
\usepackage{amssymb}
\usepackage{graphicx}
\usepackage{biblatex}

\title{Topics for Bachelor Theses}
\author{Petro Feketa}

\begin{document}
\maketitle

\section{Group chase and escape}
\subsection*{Description}
We consider two groups of agents: chasers and escapees. Chasers aim to catch all the escapees in a minimal time. The maximal velocities of the chasers are less than the maximal velocities of the escapees. Hence, the catch is only possible by a coordinated movement strategy of the chasers.
\subsection*{Tasks}
\begin{itemize}
\item to develop rules for local interaction between chasers in order to reduce the total time of the chase (model-based or machine learning approaches are possible);
\item to study the influence of communication abilities between agents on the total time of the chase.
\end{itemize}

\begin{thebibliography}{99}
\bibitem{} Kamimura, Atsushi, and Toru Ohira. "Group chase and escape." \emph{New Journal of Physics} 12.5 (2010): 053013.

\bibitem{} Vicsek, Tamás, and Anna Zafeiris. "Collective motion." \emph{Physics Reports} 517.3-4 (2012): 71-140.

\bibitem{} Mnih, Volodymyr, et al. "Human-level control through deep reinforcement learning." \emph{Nature} 518.7540 (2015): 529.

\bibitem{} Janosov, Milán, et al. "Group chasing tactics: how to catch a faster prey." \emph{New Journal of Physics} 19.5 (2017): 053003.

\end{thebibliography}


\section{Funnel control for impulsive systems}

\subsection*{Description}
A short description here. Maybe a picture of a funnel.
\subsection*{Tasks}
\begin{itemize}
\item Task 1;
\item Task 2.
\end{itemize}

\begin{thebibliography}{99}
\bibitem{} Ilchmann, A., Ryan, E., and Sangwin, C. (2002). Tracking with prescribed transient behaviour. \emph{ESAIM: Control, Optimisation and Calculus of Variations}, 7, 471-493. doi:10.1051/cocv:2002064

\end{thebibliography}

\end{document}